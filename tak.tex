\part{Den danske udgave}

\section*{Om den danske udgave}

Denne danske version af 21 Lektioner er blevet oversat af et team af
danske Bitcoin-entusiaster, der er passionerede om at sprede budskabet
om Bitcoin og dens potentiale til at forbedre verden.

Vi håber, at du har nydt at læse 21 Lessons og at du vil være med til at
sprede budskabet.

\section*{Motivation}

Vi har oversat 21 Lektioner til dansk, da vi tror på, at den tilbyder en
lettilgængelig introduktion til Bitcoin og dens principper. I modsætning
til mange andre bøger om Bitcoin dækker 21 Lektioner et bredt spektrum
af de elementer, der gør Bitcoin til Bitcoin. Den er kortfattet, skrevet
på et letforståeligt dansk og med et begrænset brug af teknisk jargon.
Vi har alle startet samme sted, hvor vi kigger på den hvide kanin, der
hopper af sted, og som i Alice i Eventyrland så starter rejsen hvor
Alice hopper ned i kaninhullet uden at tænke på hvorfor en kanin har et
lommeur og vest. Eventyret ville dog have været anderledes, hvis Alice
havde haft en rejseberetning med skrevet af Bob eller Gigi.
\newpage
\section*{Handling}

Vi opfordrer dig til at lære mere om Bitcoin og blive involveret i
Bitcoin-fællesskabet. Du kan finde mere information på følgende
ressourcer:

\begin{itemize}
\item \textbf{Telegram-kanalen "EnOgTyve":} Chat for danske Bitcoin-entusiaster hvor alle der vil lære mere om Bitcoin er velkomne. Link: \href{https://t.me/enogtyvedk}{\texttt{t.me/enogtyvedk}}.
\item \textbf{EnOgTyve's portal:} Dansksproget portal med masser af information om Bitcoin. Link: \href{https://www.enogtyve.org/}{\texttt{www.enogtyve.org}}.
\item \textbf{Bogen \textit{Bitcoinstandarden}:} En detaljeret økonomisk analyse af Bitcoin, gode penges egenskaber og af de samfundsproblemer fiat-valutaer har skabt. \newline Link: \href{https://www.bitcoinstandarden.dk/}{\texttt{www.bitcoinstandarden.dk}}.
\item \textbf{Bitcoinskolen.net:} Bitcoinskolens mission er at informere om og undervise dig i bitcoin, så du selv kan træffe dine egne valg på et oplyst grundlag. \href{https://www.bitcoinskolen.net/}{\texttt{www.bitcoinskolen.net}}.
\end{itemize}

\section*{Taknemmelighed}

Teamet vil gerne takke følgende personer for deres bidrag til
oversættelsen og korrekturlæsning:

\begin{itemize}
\item Peter Isaksen (plus hans hustru), redaktør
\item Rasmus Hansen
\item Pierre Vendelbo
\item Btcblot
\item Rune Kristensen
\end{itemize}

\textbf{Følg den hvide kanin!}
